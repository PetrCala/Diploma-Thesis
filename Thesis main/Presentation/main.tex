\documentclass{beamer} %[compress]
% alternatively add handout or draft option
%\mode<handout>%{\setbeamercolor{background canvas}{bg=black!5}}
%\usepackage{pgfpages}\\\
%\pgfpagesuselayout{4 on 1}[a4paper,border shrink=5mm,landscape]
% For handout printing, 4 slides on 1 page

\mode<presentation>
{
  \usetheme{Warsaw} % plain template Pittsburgh
  \useoutertheme{infolines} % bottom (author, title, place)
  \useoutertheme[subsection=false]{miniframes} % top (sections, frames)
  \usecolortheme{default} %blue color whale
	\setbeamertemplate{frametitle}[default][right] % frame titles on the right
  \setbeamercovered{transparent}
  % or whatever (possibly just delete it)
}

\setbeamersize{text margin left=1cm, text margin right=1cm}
% %\setbeamertemplate{navigation symbols}{}
\usepackage{babel}
\usepackage{booktabs}
\usepackage{comment}
\usepackage{multicol}
\usepackage{tikz} % Triangle
\usepackage{amssymb} % Tick marks (yes, no)
\usepackage{graphicx}
\usepackage{forest}

\usepackage[cp1250]{inputenc}
\usepackage{subfigure} %subfigures

\usepackage{times}
%\usepackage{breqn}
\usepackage{amsmath}
\usepackage[T1]{fontenc}
\def\sym#1{\ifmmode^{#1}\else\(^{#1}\)\fi} %shortcut for Stata tables
% % Or whatever. Note that the encoding and the font should match. If T1
% % does not look nice, try deleting the line with the fontenc.

% For colored circles in the economic significance frame
\newcommand{\colorcircle}[1]{
    \begin{tikzpicture}
        \pgfmathparse{ifthenelse(#1<0,"red","green")}
        \fill[\pgfmathresult] (0,0) circle (0.8mm);
    \end{tikzpicture}
}


% \def \date {June 19, 2024}													

% \logo{\includegraphics[height=0.5cm]{logo.pdf}} % logo on every page

% \date[September 23, 2023]


\title[Ability bias and education] % (optional, use only with long paper titles)
{Ability bias in the returns to schooling:}

\subtitle{How large it is and why it matters}

\author {Petr~\v{C}ala}
% - Give the names in the same order as the appear in the paper.
% - Use the \inst{?} command only if the authors have different
%   affiliation.

\institute[CUNI]
{
  Institute of Economic Studies\\
  Charles University, Prague\\
 
  \vspace{1.5em}

  \pgfdeclareimage[height=1.5cm]{logo}{Figures/logo.pdf} % logo only on title page
  \pgfuseimage{logo}

}

\date[June 19, 2024]

% \subject{Intertemporal Substitution}

% \begin{comment}
% \AtBeginSection[]
% {
%   \begin{frame}<beamer>{Outline}
%     \tableofcontents[currentsection]%,currentsubsection]
%   \end{frame}
% }
% \end{comment}

\begin{document}

\begin{frame}
    \titlepage
\end{frame}

% \begin{comment}

% \begin{frame}{Outline}
%     \tableofcontents
%     % You might wish to add the option [pausesections]
% \end{frame}

% \end{comment}

\begin{frame}{Introduction}

    \begin{large}
        Motivation
    \end{large}

    \begin{itemize}
        \item<1-> Does ability bias affect the estimation of returns to education?
        \item<2-> Two extensive meta-analyses on the topic\\(1754 and 293 observations)
    \end{itemize}

    \begin{large}
        Findings
    \end{large}

    \begin{itemize}
        \item<3-> Average effect of returns to education of around 7\%
        \item<4-> Drops by around one percentage point after correcting for publication bias
        \item <5-> Ability matters, and controlling for it in a regression decreases the expected returns to education
        \item <6-> The returns drop even further for twin studies with identical inherent ability (4\% to 6\%)
    \end{itemize}

\end{frame}

\section{Introduction}
\subsection{}

\begin{frame}{What Is Ability Bbias}

    \begin{block}<1->{Mincer Equation (Mincer, 1974)}
        \large
        \begin{equation*}
            \text{Wage} \sim \text{Schooling} + \text{Experience} + \text{Experience}^2
        \end{equation*}
    \end{block}

    \begin{itemize}
        \item<1-> Returns to education: The increase in earnings due to an additional year of schooling
        \item<2-> Ability bias: Distorted estimation of returns to education due to omission of ability (Blackburn \& Neumark, 1993)
        \item<3-> Ability correlates with both education and earnings
        \item<4-> Sorting bias: Correlation between ability and education
        \item<5-> How to separate the effect of education from the effect of ability?
    \end{itemize}

\end{frame}

\begin{frame}{Ways To Deal With Ability Bias}

    \begin{itemize}
        \item<1-> Inclusion of Ability Measures
              \begin{itemize}
                  \item<2-> Use cognitive test scores as control variables
                  \item<2-> Separates effect of education from ability
              \end{itemize}
        \item<3-> Instrumental Variables (IV)
              \begin{itemize}
                  \item<4-> Find variable correlated with education, not with error term
                  \item<4-> Isolates exogenous variation in education
              \end{itemize}
        \item<5-> Sibling and Twin Studies
              \begin{itemize}
                  \item<6-> Compare siblings/twins with different education levels
                  \item<6-> Controls for family and genetic factors
              \end{itemize}
        \item<7-> Other Methods
              \begin{itemize}
                  \item<8-> Fixed Effects Models
                  \item<8-> Nonparametric methods
              \end{itemize}
    \end{itemize}

\end{frame}

\begin{frame}{What do we already know?}
    \begin{table}[!t]
        \centering
        \footnotesize
        \begin{tabular}{
                @{}
                l
                *{5}{c}
                @{}}
            \toprule
            \textbf{Study name}               & \textbf{AB} & \textbf{AB*} & \textbf{PB} & \textbf{PB*} & \textbf{Method} \\
            \midrule
            Psacharopoulos (1994)             & .           & .            & .           & .            & .               \\
            Fleisher et al. (2005)            & .           & .            & .           & .            & \checkmark      \\
            Churchill \& Mishra (2018)        & .           & .            & \checkmark  & \checkmark   & \checkmark      \\
            Psacharopoulos \& Patrinos (2018) & .           & .            & .           & .            & .               \\
            Patrinos \& Psacharopoulos (2020) & .           & .            & .           & .            & .               \\
            Cui \& Martins (2021)             & .           & .            & \checkmark  & \checkmark   & \checkmark      \\
            Iwasaki \& Ma (2021)              & .           & .            & \checkmark  & .            & \checkmark      \\
            Ma \& Iwasaki (2021)              & .           & .            & \checkmark  & \checkmark   & \checkmark      \\
            Wincenciak et al. (2022)          & \checkmark  & \checkmark   & .           & .            & \checkmark      \\
            Horie \& Iwasaki (2023)           & .           & .            & \checkmark  & .            & .               \\
            \midrule
            Number of studies:                & 1           & 1            & 5           & 3            & 6               \\
            Percentage of studies:            & 10\%        & 10\%         & 50\%        & 30\%         & 60\%            \\
            \bottomrule
        \end{tabular}
    \end{table}
\end{frame}



\section{Contribution}
\subsection{}

\begin{frame}{My contribution}

    \begin{itemize}
        \item<1-> A large meta-analysis of 1754 estimates of returns to education over 115 studies
        \item<2-> Correct for publication bias, observe heterogeneity
        \item<3-> Observe the isolated effect of ability
        \item<4-> Conduct a whole another meta-analysis comprised of twin studies (293 observations)
        \item<5-> Fully automate the whole analysis process
    \end{itemize}

\end{frame}


% \begin{frame}{Different Ways To Address Ability}

%     \begin{itemize}
%         \item Google Scholar search
%         \item 574 records identified through query
%         \item 200 of them screened and evaluated for eligibility
%         \item 74 fulfilled selection criteria
%         \item 41 more studies added through snowballing
%         \item 115 final studies yielded a total of 1754 estimates
%     \end{itemize}
% \end{frame}

\begin{frame}{Different Approach to Ability}

    Four ways to address ability:
    \vspace{0.3cm}

    \begin{itemize}
        \item<1-> Directly - using cognitive test scores or proxies thereof
        \item<2-> Indirectly - using instrumental variables or other methods
        \item<3-> Verbally - acknowledging the issue
        \item<4-> Not at all - ignoring the problem
    \end{itemize}

\end{frame}



\begin{frame}{Estimates of ability across the dataset}

    \begin{center}
        \includegraphics[width=0.8\textwidth]{Figures/prima_facie_ability.png}
    \end{center}

\end{frame}



\begin{frame}{Graphical Test Using a Funnel Plot}
    \begin{figure}[htbp]
        \begin{center}
            \includegraphics[width=0.7\textwidth]{Figures/funnel.png}
        \end{center}
    \end{figure}
\end{frame}




\begin{frame}{Statistical Tests and Publication Bias}
    \begin{tiny}

        \begin{table}[!t]
            \centering
            \begin{tabular}{
                @{}l*{6}{c}
                } %one left column, five center (*{} makes the cols inherit attributes)
                \toprule
                \multicolumn{1}{l}{}                  &
                \multicolumn{1}{c}{\textbf{OLS}}      &
                \multicolumn{1}{c}{\textbf{FE}}       &
                \multicolumn{1}{c}{\textbf{BE}}       &
                \multicolumn{1}{c}{\textbf{RE}}       &
                \multicolumn{1}{c}{\textbf{Study}}    &
                \multicolumn{1}{c}{\textbf{Precision}}                                                              \\
                \midrule
                Publication bias                      & 0.832   & 0.746   & 0.752   & 0.747   & 1.169     & 0.262   \\
                \emph{\hspace{0.2cm}(Standard error)} & (0.097) & (0.060) & (0.244) & (0.058) & (0.121)   & (0.425) \\
                \addlinespace[0.5em]
                Effect beyond bias                    & 6.408   & 6.517   & 6.741   & 6.708   & 6.294     & 6.540   \\
                \emph{\hspace{0.2cm}(Constant)}       & (0.118) & (0.107) & (0.418) & (0.294) & (0.153)   & (0.168) \\
                \addlinespace[0.5em]
                \toprule
                \addlinespace[0.5em]
                \multicolumn{1}{c}{}                  &
                \multicolumn{1}{c}{\textbf{WAAP}}     &
                \multicolumn{1}{c}{\textbf{Top10}}    &
                \multicolumn{1}{c}{\textbf{Stem}}     &
                \multicolumn{1}{c}{\textbf{Hier}}     &
                \multicolumn{1}{c}{\textbf{AK}}       &
                \multicolumn{1}{c}{\textbf{Kink}}                                                                   \\
                \midrule
                Publication bias                      &         &         &         & 0.503   & P = 2.764 & 0.262   \\
                                                      &         &         &         & (0.168) & (0.107)   & (0.39)  \\
                \addlinespace[0.5em]
                Effect beyond bias                    & 6.9     & 6.439   & 7.2     & 6.801   & 6.548     & 6.54    \\
                                                      & (0.092) & (0.548) & (1.186) & (0.266) & (0.091)   & (0.054) \\
                \addlinespace[0.5em]
                \midrule
                Observations                          & 1,754   & 1,754   & 1,754   & 1,754   & 1,754     & 1,754   \\

                \bottomrule
            \end{tabular}
        \end{table}

    \end{tiny}
\end{frame}






\begin{frame}{Individual Variables in Returns to Education}

    Over 30 variables split into six categories:
    \vspace{0.3cm}

    \begin{itemize}
        \item<1-> Estimates and their descriptive statistics
        \item<2-> Estimate characteristics
        \item<3-> Data characteristics
        \item<4-> Spatial/structural variation
        \item<5-> Estimation method
        \item<6-> Publication characteristics
    \end{itemize}

\end{frame}


\begin{frame}{Model Inclusion in Bayesian Model Averaging}
    \begin{center}
        \includegraphics[width=0.7\textwidth]{Figures/bma_UIP_dilut_results.png}
    \end{center}
\end{frame}


\begin{frame}{Economic Significance of Key Variables}


    \begin{tiny}
        \begin{table}[!htbp]
            \begin{tabular}{l*{4}{c}}
                \toprule
                                                          & \multicolumn{2}{c}{One SD change} & \multicolumn{2}{c}{Maximum change}                                \\
                                                          & Effect on Returns                 & \% of BP                           & Effect on Returns & \% of BP \\
                \midrule
                \colorcircle{0.642} Standard Error        & 0.642                             & 9.82\%                             & 3.435             & 52.56\%  \\
                \colorcircle{-0.428} Estimate: Sub-region & -0.428                            & -6.55\%                            & -1.433            & -21.92\% \\
                \colorcircle{-0.612} Estimate: Region     & -0.612                            & -9.37\%                            & -1.325            & -20.27\% \\
                \colorcircle{0.56} Education: Years       & 0.566                             & 8.67\%                             & 1.175             & 17.98\%  \\
                \colorcircle{-0.40} Wage: Log Daily       & -0.405                            & -6.2\%                             & -1.384            & -21.18\% \\
                \colorcircle{0.53} Micro Data             & 0.532                             & 8.13\%                             & 1.391             & 21.29\%  \\
                \colorcircle{0.53} Primary Education      & 0.535                             & 8.18\%                             & 3.540             & 54.16\%  \\
                \colorcircle{1.366} Higher Education      & 1.366                             & 20.91\%                            & 5.521             & 84.48\%  \\
                \colorcircle{-0.42} Gender: Male          & -0.425                            & -6.5\%                             & -1.215            & -18.58\% \\
                \colorcircle{-0.60} Ethnicity: Caucasian  & -0.608                            & -9.3\%                             & -1.449            & -22.18\% \\
                \colorcircle{0.43} Method: 2SLS           & 0.433                             & 6.62\%                             & 1.474             & 22.56\%  \\
                \colorcircle{0.824} Method: IV            & 0.824                             & 12.61\%                            & 2.627             & 40.2\%   \\
                \colorcircle{-0.388} Ability: Direct      & -0.388                            & -5.94\%                            & -1.138            & -17.41\% \\
                \colorcircle{0.27} Ability: Uncontrolled  & 0.271                             & 4.15\%                             & 0.548             & 8.39\%   \\
                \colorcircle{-0.89} Control: Age          & -0.895                            & -13.69\%                           & -1.883            & -28.81\% \\
                \colorcircle{1.315} Control: Age$^2$      & 1.315                             & 20.12\%                            & 2.945             & 45.06\%  \\
                \colorcircle{0.878} Control: Area         & 0.878                             & 13.44\%                            & 1.781             & 27.24\%  \\
                \colorcircle{-0.296} Impact Factor        & -0.296                            & -4.53\%                            & -1.349            & -20.64\% \\
                \colorcircle{-0.44} Study: Published      & -0.445                            & -6.8\%                             & -1.047            & -16.01\% \\
                \bottomrule
            \end{tabular}
        \end{table}
    \end{tiny}
\end{frame}



\section{Twins}
\subsection{}



\begin{frame}{Making a twin dataset}

    \begin{itemize}
        \item Only subjects with identical inherent ability - twins
        \item 16 twin studies with 293 observations
        \item Assumption: Differences in returns to education are due to differences in education
    \end{itemize}

\end{frame}


\begin{frame}{Twin Funnel Plot}
    \begin{figure}[htbp]
        \begin{center}
            \includegraphics[width=0.7\textwidth]{Figures/funnel_twins.png}
        \end{center}
    \end{figure}
\end{frame}



\begin{frame}{Publication bias for twins}
    \begin{tiny}
        \begin{table}[!htbp]
            \begin{tabular}{
                    @{}
                    l*{6}{c}} %one left column, five center (*{} makes the cols inherit attributes)
                \toprule
                \multicolumn{1}{c}{}                  &
                \textbf{OLS}                          &
                \textbf{FE}                           &
                \textbf{BE}                           &
                \textbf{RE}                           &
                \textbf{Study}                        &
                \textbf{Precision}                                                                                \\
                \midrule

                Publication bias                      & 1.347   & 0.602   & 2.133   & 0.840   & 0.947   & 2.897   \\
                \emph{\hspace{0.2cm}(Standard error)} & (0.138) & (0.162) & (0.505) & (0.154) & (0.177) & (0.442) \\
                \addlinespace[0.5em]
                Effect beyond bias                    & 4.735   & 5.574   & 4.106   & 5.55    & 4.754   & 3.907   \\
                \emph{\hspace{0.2cm}(Constant)}       & (0.175) & (0.219) & (0.711) & (0.342) & (0.185) & (0.232) \\

                \midrule

                                                      &
                \textbf{WAAP}                         &
                \textbf{Top10}                        &
                \textbf{Stem}                         &
                \textbf{Hier}                         &
                \textbf{AK}                           &
                \textbf{Kink}                                                                                     \\
                \midrule
                Publication bias                      &         &         &         & 0.601   & 2.257   & 2.895   \\
                                                      &         &         &         & (0.365) & (0.126) & (0.435) \\
                \addlinespace[0.5em]
                Effect beyond bias                    & 5.77    & 4.314   & 3.403   & 5.857   & 5.616   & 3.908   \\
                                                      & (0.159) & (0.265) & (0.95)  & (0.544) & (0.157) & (0.093) \\

                \midrule
                \addlinespace[0.5em]
                Observations                          & 293     & 293     & 293     & 293     & 293     & 293     \\

                \bottomrule
            \end{tabular}
        \end{table}

    \end{tiny}
\end{frame}


\section{Conclusion}
\subsection{}


\begin{frame}{Conclusion}
    \begin{itemize}
        \item<1-> An overall effect of returns to schooling drops roughly one percentage point (7\% to 6\%) after corrected for publication bias
        \item<2-> Ability matters, and controlling for it in the regression decreases the expected returns to schooling
        \item<3-> Nine variables have a significant positive influence on returns to schooling, while ten have a negative one
        \item<4-> The returns to schooling drop even further for twin studies with identical inherent ability (4\% to 6\%)
    \end{itemize}
\end{frame}

% \section{Extensions}
% \subsection{}

% \begin{frame}{There's more...}

%   \begin{center}
%     \begin{LARGE}
%       Technology
%     \end{LARGE}
%   \end{center}

% \end{frame}

% \begin{frame}{Stages of Meta-Analysis}
%   \centering
%   \Large
%   \begin{tikzpicture}[node distance=1.3cm]
%     \node (idea) {Idea};
%     \node (data) [below of=idea] {Data};
%     \node (analysis) [below of=data] {Analysis};
%     \node (writing) [below of=analysis] {Writing};
%     \node (results) [below of=writing] {Results};

%     \draw[->] (idea) -- (data);
%     \draw[->] (data) -- (analysis);
%     \draw[->] (analysis) -- (writing);
%     \draw[->] (writing) -- (results);

%     \only<2>{\draw[red,thick] (analysis.south west) -- (analysis.north east);}
%   \end{tikzpicture}
% \end{frame}




% All of the following is optional and typically not needed. 
\appendix
\section{Appendix}
% \section<presentation>*{\appendixname}


\begin{frame}
    \frametitle<presentation>{}

    \begin{center}
        \begin{LARGE}
            Thank you!
        \end{LARGE}
    \end{center}

\end{frame}


\begin{frame}%[allowframebreaks]
    \frametitle<presentation>{References}

    \begin{thebibliography}{10}

        %         %\beamertemplatebookbibitems
        %         % Start with overview books.


        %         \beamertemplatearticlebibitems
        %         % Followed by interesting articles. Keep the list short. 

        \bibitem[{Mincer (1974)}]{Mincer1974}
        Mincer, Jacob A. "The human capital earnings function."
        \newblock\emph{Schooling, experience, and earnings, pp. 83-96. NBER, 1974.}

        \bibitem[{Blackbuck \& Neumark (1993)}]{Blackburn1993}
        Blackburn, McKinley L., and David Neumark. "Omitted-ability bias and the increase in the return to schooling." \newblock\emph{Journal of labor economics 11, no. 3 (1993): 521-544}.


    \end{thebibliography}
\end{frame}

% \begin{frame}{Graphing out the individual method differences}
%     \begin{center}
%         \includegraphics[width=0.7\textwidth]{Figures/twins_prima_method.png}
%     \end{center}
% \end{frame}



\begin{frame}{Schooling in Years vs. Levels}
    \begin{center}

        \begin{Large}
            $
                S_i = \left(1 + \beta_{i, higher} - \beta_{i, lower}\right)^{\frac{1}{Y_{i, higher} - Y_{i, lower}}} - 1
            $
        \end{Large}


    \end{center}
\end{frame}

\begin{frame}{Results Using Some Recent Methods}
    \begin{tiny}

        \begin{table}[!b]
            \begin{tabular}{
                    l*{4}{c}
                }
                \toprule

                \multicolumn{5}{l}{\textit{Panel A: p-hacking tests by Elliott et al. (2022)}}                        \\
                \addlinespace[0.3em]
                                                                &
                \textbf{Non-increas.}                           & \textbf{Monotonicity} &         &                   \\
                \midrule
                Non-increas.                                    & 0.819                 & 0.871   &         &         \\
                Observations (p$\leq$0.1)                       & 1,610                 & 1,610   &         &         \\
                Observations                                    & 1,754                 & 1,754   &         &         \\
                \addlinespace[0.1em]
                \hline

                \addlinespace[0.5em]
                \multicolumn{5}{l}{\textit{Panel B: MAIVE estimator (Irsova et al., 2023)}}                           \\
                \addlinespace[0.3em]
                                                                & \textbf{Results}      &         &         &         \\

                \midrule
                MAIVE coefficient                               & 5.736                 &         &         &         \\
                Standard Error                                  & (0.460)               &         &         &         \\
                F-test                                          & 12.491                &         &         &         \\
                Observations                                    & 1,754                 &         &         &         \\
                \addlinespace[0.1em]
                \hline

                \addlinespace[0.5em]
                \multicolumn{5}{l}{\textit{Panel C: Robust Bayesian Model Averaging (Bartos et al., 2022)}}           \\
                \addlinespace[0.3em]
                                                                &
                \multicolumn{1}{c}{\centering{\textbf{Mean}}}   &
                \multicolumn{1}{c}{\centering{\textbf{Median}}} &
                \multicolumn{1}{c}{\centering{\textbf{0.025}}}  &
                \multicolumn{1}{c}{\centering{\textbf{0.975}}}                                                        \\
                \midrule
                Coefficient                                     & 7.125                 & 7.124   & 6.946   & 7.299   \\
                Standard Error                                  & (3.505)               & (3.504) & (3.371) & (3.645) \\
                Observations                                    & 1,754                 & 1,754   & 1,754   & 1,754   \\
                \addlinespace[0.1em]
                \bottomrule
            \end{tabular}
        \end{table}

    \end{tiny}
\end{frame}

\begin{frame}{Best-Practice Estimate Across Literature}

    \begin{table}[!htbp]
        \scriptsize
        \begin{tabular}{l*{3}{c} }
            \toprule
            \textbf{Study} & \textbf{Estimate} & \textbf{95\% Confidence Interval} & \textbf{Studies} \\
            \midrule
            Author         & 6.536             & (5.762; 7.310)                    & 0                \\
            Query          & 7.529             & (3.552; 11.506)                   & 74               \\
            Snowballing    & 6.346             & (2.530; 10.162)                   & 41               \\
            All studies    & 7.109             & (3.046; 11.17)                    & 115              \\
            \bottomrule
        \end{tabular}
    \end{table}

\end{frame}

\begin{frame}{Aggregating BPE Results}
    \begin{center}
        \includegraphics[width=0.8\textwidth]{Figures/bpe_ability.png}
    \end{center}
\end{frame}


\begin{frame}{Meta-Analysis Automatization Script}

    \begin{itemize}
        \item Meta-analysis, but automatized
        \item One script and a bit of parametrization
        \item Faster methods, all ran locally
        \item Caches and file handling
        \item All results calculated, formatted, and exported within minutes
    \end{itemize}

\end{frame}


\begin{frame}{Project structure}

    \begin{flushleft}
        \includegraphics[width=0.5\textwidth]{Figures/project_structure.png}
    \end{flushleft}

\end{frame}

% \begin{frame}[fragile]{Project structure}

%     \begin{scriptsize}
%         \begin{forest}
%             for tree={
%             grow'=0,
%             child anchor=west,
%             parent anchor=south,
%             anchor=west,
%             calign=first,
%             edge path={
%                     \noexpand\path [draw, \forestoption{edge}]
%                     (!u.south west) +(7.5pt,0) |- (.child anchor) pic {folder} \forestoption{edge label};
%                 },
%             font=\ttfamily,
%             before typesetting nodes={
%                     if n=1
%                         {insert before={[,phantom]}}
%                         {}
%                 },
%             fit=band,
%             before computing xy={l=15pt},
%             }
%             [.
%             [data/ ]
%             [pckg/ ]
%             [scripts/ ]
%             [results/
%             [graphic/]
%             [numeric/]
%             [main\_results.txt]
%             ]
%             [main\_master\_thesis\_cala.R]
%             [script\_runner\_master\_thesis\_cala.R]
%             [source\_master\_thesis\_cala.R]
%             [README.md]
%             [user\_parameters.yaml]
%             ]
%         \end{forest}
%     \end{scriptsize}

% \end{frame}

\begin{frame}{Graphic Results}
    \begin{center}
        \includegraphics[width=1\textwidth]{Figures/graphical_results.png}
    \end{center}
\end{frame}


\begin{frame}{Numeric Results}
    \begin{center}
        \includegraphics[width=0.4\textwidth]{Figures/numeric_results.png}
    \end{center}
\end{frame}

\begin{frame}{main\_results.txt}
    \begin{center}
        \includegraphics[width=1\textwidth]{Figures/main_results.png}
    \end{center}
\end{frame}


\begin{frame}{It can do this...}
    \begin{itemize}
        \item Variable summary statistics
        \item Effect summary statistics
        \item Prima Facie graphs
        \item Box plot
        \item Funnel plot
        \item T-statistic histogram
        \item Linear tests
              \begin{itemize}
                  \item OLS
                  \item Between Effects
                  \item Fixed Effects
                  \item Random Effects
                  \item Study-weighted OLS
                  \item Precision-weighted OLS
              \end{itemize}
    \end{itemize}
\end{frame}



\begin{frame}{...and this...}
    \begin{itemize}

        \item Non-linear tests
              \begin{itemize}
                  \item Weighted Average of Adequately Powered
                  \item Top10
                  \item Stem-based method
                  \item Hierarchial Bayes
                  \item Selection model
                  \item Endogenous Kink model
              \end{itemize}
        \item Tests relaxing exogeneity
              \begin{itemize}
                  \item Instrumental Variable regression
                  \item p-uniform*
              \end{itemize}
        \item P-hacking tests
              \begin{itemize}
                  \item Caliper tests
                  \item Elliott tests
                  \item MAIVE estimator
              \end{itemize}

    \end{itemize}
\end{frame}

\begin{frame}{...and even this!}
    \begin{itemize}


        \item Bayesian Model Averaging
        \item Frequentist Model Averaging
        \item Model Averaging variables description table
        \item Best-practice estimate
        \item Best-practice estimate: Graphs
        \item Best-practice estimate: Summary statistics
        \item Robust Bayesian Model Averaging
    \end{itemize}
\end{frame}


\begin{frame}{Available on GitHub}

    \begin{center}
        \includegraphics[width=0.08\textwidth]{Figures/github.png}

        \vspace{0.5cm}

        \begin{Large}
            github.com/PetrCala/Diploma-Thesis \\
        \end{Large}

    \end{center}

\end{frame}


\end{document}



