%-----<<<<<<<<<<<< START >>>>>>>>>>>>-----
\documentclass [a4paper,12pt]{report}  
\usepackage[a-2u]{pdfx}      		
\usepackage[cp1250]{inputenc}													
\usepackage{lmodern}
\usepackage[T1]{fontenc}
\usepackage{textcomp}
\usepackage{setspace}
\onehalfspacing

\begin{document}

\section*{Abstrakt}

Literatura o n\'{a}vratnosti \v{s}kolstv\'{i} pat\v{r}\'{i} mezi nejrozs\'{a}hlej\v{s}\'{i} a nejd\.{u}kladn\v{e}ji zkouman\'{e} oblasti v ekonomick\'{e} teorii, p\v{r}esto v\v{e}nuje velmi m\'{a}lo pozornosti probl\'{e}mu dovednostn\'{i}ho zkreslen\'{i}. Dosud neexistuje shoda ohledn\v{e} m\'{i}ry, do jak\'{e} lidsk\'{a} inteligence a schopnost ovliv\v{n}uje n\'{a}vratnost do \v{s}kolstv\'{i}. V t\'{e}to pr\'{a}ci sestavuji dataset \v{c}\'{i}taj\'{i}c\'{i} 1754 pozorov\'{a}n\'{i} ze 154 studi\'{i} a pokou\v{s}\'{i}m se systematicky zanalyzovat roli dovednosti v kontextu Mincerovy rovnice. Nejprve prov\v{e}\v{r}uji publika\v{c}n\'{i} zkreslen\'{i} a zji\v{s}\v{t}uji, \v{z}e jeho zohledn\v{e}n\'{i} sni\v{z}uje n\'{a}vratnost vzd\v{e}l\'{a}n\'{i} o p\v{r}ibli\v{z}n\v{e} jeden procentn\'{i} bod na zhruba 6-7\%. Pot\'{e} pomoc\'{i} bayesovsk\'{e}ho pr\.{u}m\v{e}rov\'{a}n\'{i} identifikuji 19 d\.{u}le\v{z}it\'{y}ch prom\v{e}nn\'{y}ch, kter\'{e} v\'{y}znamn\v{e} ovliv\v{n}uj\'{i} efekt, jako je nap\v{r}. typ nebo \'{u}rove\v{n} vzd\v{e}l\'{a}n\'{i}, pohlav\'{i}, nebo r\.{u}zn\'{i}c\'{i} se methoda odhadu. Nakonec prozkoum\'{a}m kontext studi\'{i} zam\v{e}\v{r}en\'{e} na dvoj\v{c}ata, kde se p\v{r}edpokl\'{a}d\'{a} identick\'{a} schopnost jedinc\.{u}. Sestavuji proto zcela nov\'{y} dataset \v{c}\'{i}taj\'{i}c\'{i} 154 odhad\.{u} z 13 studi\'{i} a zji\v{s}\v{t}uji, \v{z}e vliv vzd\v{e}l\'{a}n\'{i} kles\'{a} v tomto p\v{r}\'{i}pad\v{e} je\v{s}t\v{e} v\'{i}ce, na neuv\v{e}\v{r}iteln\'{y}ch 4-6\% n\'{a}vratnosti do vzd\v{e}l\'{a}n\'{i} za ka\v{z}d\'{y} dal\v{s}\'{i} rok ve \v{s}kolstv\'{i} str\'{a}ven\'{y}.
\end{document}