\documentclass[12pt]{article}
\usepackage[utf8]{inputenc}
% \usepackage[onehalfspacing]{setspace}
\usepackage{amsmath}
\usepackage{geometry}
\geometry{a4paper}

\begin{document}

\begin{center}
    \large{Thesis Defense Handout \\  \vspace{0.5cm}\Large \textit{Ability bias in the returns to schooling:\\\smallskip How large it is and why it matters}}\\
    \vspace{0.2cm}
    \normalsize{\textit{by}\\}
    \vspace{0.1cm}
    \large{Petr \v{C}ala}
\end{center}
% 

\section*{Importance}
Ability is potentially a highly influential factor in the Mincer equation, on of the all-times most prominent models in labor economics.
\begin{itemize}
    \item Labor policy: If ability is not controlled for, the returns to schooling are overestimated, which may lead to suboptimal policy decisions.
    \item Impact on the individual: For individuals, knowing, or contrarily the lack of the true extent to which one's future successess are determined by ability is inherently tied to making optimal life decisions.
\end{itemize}

\section*{Problem Formulation}

The influence and extent of ability bias in the returns to schooling has long been a disputed topic in the returns to schooling literature.

\begin{itemize}
    \item Out of 10 existing meta-analyses, only a single one searches for ability bias. That study suggests that ability is a significant factor in the returns to schooling.
    \item On the other hand, many researchers in the field dismiss the influence of general ability on returns to schooling completely, leading a dissatisfactory state of the literature on one of the most fundamental questions in labor economics.

\end{itemize}
\clearpage

\section*{Methodology \& Data}

Using a battery of standard and cutting-edge meta-analytical techniques to synthesize the existing literature into a comprehensive analysis.

\begin{itemize}
    \item I assemble two full datasets on which I run my analyses: one comprised of 1754 estimates over 174 studies, and another one, focused only on identical twins, with 154 estimates from 13 studies.
    \item I test for publication bias, within-study heterogeneity, construct a best-practice estimate for all available studies, calculate economic significance, and run all of this on the twin dataset as well.
\end{itemize}

\section*{Contribution}

\begin{itemize}
    \item The thesis brings focus to a rather neglected question of ability bias, claiming that ability directly and negatively affects the estimates of returns to schooling, among a myriad of other variables.
    \item The thesis also runs another, complete meta-analysis comprised only of natural twin studies, suggesting that the returns to schooling drop by up to 2-3 percentage point when ability is controlled for.
    \item The work is backuped up by an open-source project that allows anyone to run a full-fledged meta-analysis using their own data in a matter of minutes.

\end{itemize}

\section*{Response to Comments on the Thesis}
\textit{TBA}

\end{document}
