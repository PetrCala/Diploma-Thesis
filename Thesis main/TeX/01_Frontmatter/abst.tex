\section*{Abstract}

The literature on returns to schooling is among the largest and most scrutinised fields in economics, yet it pays so little attention to the issue of ability bias. To this date, there is no consensus on the extent to which general intelligence influences the estimated returns to schooling. In this work, I assemble a dataset containing 1754 from 154 studies, and attempt to systematically analyse the role of ability in the context of the Mincer equation. I first check for publication bias, and find that controlling for it lowers the returns to education by roughly one percentage point to roughly 6-7\%. Then, using model averaging, I identify 19 highly relevant variables that influence the effect, including education type, education level, gender, or the estimation method. Lastly, in the context of twin studies  where ability is assumed identical, I construct a new dataset comprised of 154 estimates across 13 studies, and find that the influence of education drops even further, to a staggering 4-6\% returns to an additional year of schooling.



\bigskip

\begin{tabular}{lp{8.6cm}}
	\textbf{JEL Classification}  & \JEL                                        \\
	\textbf{Keywords}            & \Keywords                                   \\
	                             &                                             \\
	\textbf{Title}               & \Bookname                                   \\
	\textbf{Author's e-mail}     & \texttt{\href{mailto:\Email}{\Email}}       \\
	\textbf{Supervisor's e-mail} & \texttt{\href{mailto:\EmailSup}{\EmailSup}} \\
\end{tabular}

\clearpage

\section*{Abstrakt}\label{abstract}

Literatura o n\'{a}vratnosti \v{s}kolstv\'{i} pat\v{r}\'{i} mezi nejrozs\'{a}hlej\v{s}\'{i} a nejd\.{u}kladn\v{e}ji zkouman\'{e} oblasti v ekonomick\'{e} teorii, p\v{r}esto v\v{e}nuje velmi m\'{a}lo pozornosti probl\'{e}mu dovednostn\'{i}ho zkreslen\'{i}. Dosud neexistuje shoda ohledn\v{e} m\'{i}ry, do jak\'{e} lidsk\'{a} inteligence a schopnost ovliv\v{n}uje n\'{a}vratnost do \v{s}kolstv\'{i}. V t\'{e}to pr\'{a}ci sestavuji dataset \v{c}\'{i}taj\'{i}c\'{i} 1754 pozorov\'{a}n\'{i} ze 154 studi\'{i} a pokou\v{s}\'{i}m se systematicky zanalyzovat roli dovednosti v kontextu Mincerovy rovnice. Nejprve prov\v{e}\v{r}uji publika\v{c}n\'{i} zkreslen\'{i} a zji\v{s}\v{t}uji, \v{z}e jeho zohledn\v{e}n\'{i} sni\v{z}uje n\'{a}vratnost vzd\v{e}l\'{a}n\'{i} o p\v{r}ibli\v{z}n\v{e} jeden procentn\'{i} bod na zhruba 6-7\%. Pot\'{e} pomoc\'{i} bayesovsk\'{e}ho pr\.{u}m\v{e}rov\'{a}n\'{i} identifikuji 19 d\.{u}le\v{z}it\'{y}ch prom\v{e}nn\'{y}ch, kter\'{e} v\'{y}znamn\v{e} ovliv\v{n}uj\'{i} efekt, jako je nap\v{r}. typ nebo \'{u}rove\v{n} vzd\v{e}l\'{a}n\'{i}, pohlav\'{i}, nebo r\.{u}zn\'{i}c\'{i} se methoda odhadu. Nakonec prozkoum\'{a}m kontext studi\'{i} zam\v{e}\v{r}en\'{e} na dvoj\v{c}ata, kde se p\v{r}edpokl\'{a}d\'{a} identick\'{a} schopnost jedinc\.{u}. Sestavuji proto zcela nov\'{y} dataset \v{c}\'{i}taj\'{i}c\'{i} 154 odhad\.{u} z 13 studi\'{i} a zji\v{s}\v{t}uji, \v{z}e vliv vzd\v{e}l\'{a}n\'{i} kles\'{a} v tomto p\v{r}\'{i}pad\v{e} je\v{s}t\v{e} v\'{i}ce, na neuv\v{e}\v{r}iteln\'{y}ch 4-6\% n\'{a}vratnosti do vzd\v{e}l\'{a}n\'{i} za ka\v{z}d\'{y} dal\v{s}\'{i} rok ve \v{s}kolstv\'{i} str\'{a}ven\'{y}.

% Literatura o návratnosti školství patří mezi nejrozsáhlejší a nejdůkladněji zkoumané oblasti v ekonomické teorii, přesto věnuje velmi málo pozornosti problému dovednostního zkreslení. Dosud neexistuje shoda ohledně míry, do jaké lidská inteligence a schopnost ovlivňuje návratnost do školství. V této práci sestavuji dataset čítající 1754 pozorování ze 154 studií a pokouším se systematicky zanalyzovat roli dovednosti v kontextu Mincerovy rovnice. Nejprve prověřuji publikační zkreslení a zjišťuji, že jeho zohlednění snižuje návratnost vzdělání o přibližně jeden procentní bod na zhruba 6-7\%. Poté pomocí bayesovského průměrování identifikuji 19 důležitých proměnných, které významně ovlivňují efekt, jako je např. typ nebo úroveň vzd��lán��, pohlaví, nebo různící se methoda odhadu. Nakonec prozkoumám kontext studií zaměřené na dvojčata, kde se předpokládá identická schopnost jedinců. Sestavuji proto zcela nový dataset čítající 154 odhadů z 13 studií a zjišťuji, že vliv vzdělání klesá v tomto případě ještě více, na neuvěřitelných 4-6\% návratnosti do vzdělání za každý další rok strávený ve školství.




\bigskip

\begin{tabular}{lp{7.7cm}}
	\textbf{Klasifikace JEL}                & \JEL                                        \\
	\textbf{Kl\'{i}\v{c}ov\'{a} slova}      & \Klic                                       \\
	                                        &                                             \\
	\textbf{N\'{a}zev pr\'{a}ce}            & \BooknameCZ                                 \\
	\textbf{E-mail autora}                  & \texttt{\href{mailto:\Email}{\Email}}       \\
	\textbf{E-mail vedouc\'{i}ho pr\'{a}ce} & \texttt{\href{mailto:\EmailSup}{\EmailSup}} \\
\end{tabular}

