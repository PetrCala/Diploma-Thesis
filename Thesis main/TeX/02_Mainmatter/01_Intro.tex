\chapter{Introduction}
\label{chap:one}

In psychology, general intelligence has long been recognized as one the most reliable predictors of one's financial and, in fact, general success in life \citep{deary2007intelligence,gottfredson1997g}. This is also true in economics, where the economic returns to schooling are often estimated to be the highest for those with the highest cognitive abilities \citep{herrnstein2010bell}. Even through revision and scrutiny, it holds that among all other factors, general intelligence, or so-called g-factor \citep{olea1994predicting}, is the most important predictor of one's life outcomes \citep{ganzach2018wages}. However, tying this predictive power to policy-making can prove rather tricky, as policy analysis requires causal effects to assess its impact - something unavailable in the predictive analysis which works with correlations.

Still, it remains a fundamental goal of policymakers to improve a person's probability to succeed in the labor market, and in doing so they rely a great deal on the predicted returns to education. In other words, how much will an investment into schooling pay off in the future for an individual. The most prominent model to estimate these returns is undeniably the Mincer equation \citep{mincer1974schooling}, which suggests that each additional year of education produces a private (i.e. individual) rate of return to schooling of about 5-8\% per year, ranging from a low of 1\% to more than 20\% in some countries \citep{psacharopoulos2018meta}. However, the Mincer equation is a rather simplistic model and thus fails to account for many existing biases in the literature, such as the ability bias, which is the tendency for the economic returns to schooling to rise among those with high ability \citep{griliches1977estimating,heckman2001identifying}. This may pose a problem, as the estimated effect of education could be biased, and the suggested returns to schooling could thus be inflated.

In the literature, the approach towards ability bias is all but unified. Some, including \cite{ashenfelter1999schooling}, pay it little heed and suggest that the bias is of little importance. Others, such as \cite{wincenciak2022meta}, claim that the bias does have a significant impact on the estimates of return to schooling, and should be accounted for. Consequently, many questions arise as to the validity of the Mincer equation, and the numerous findings presented on the topic thus far. How large is the ability bias in the literature? How much, if at all, does it affect the estimated returns to schooling? What role does it play in regard to the predictive power of general intelligence?

I hypothesize that general ability, being the most important predictor of life outcomes in psychology literature, has an important role in the economic literature as well. Almost surprisingly, I found only a single meta-analysis of the vast literature on private returns to schooling that would systematically analyze the topic of ability bias \citep{wincenciak2022meta} - not nearly enough given the importance of the topic to be considered conclusive evidence. This thesis aims to fill this gap and provide a systematic analysis of the role of ability bias in the context of the Mincer equation.

For this analysis, I first dive into the issue of publication bias - a researcher's preference to publish significant results over insignificant ones \citep{stanley2001wheat}. In an extensive dataset of 174 studies, I find mild conclusive evidence for the presence of this bias, after treating for which the returns to education drop by less than a percentage point. Next, I focus on heterogeneity and the influence of individual variables on the overall effect. Along with identifying several variables with negative influence (e.g., region type, study publication status), and others with a positive one (e.g., higher education, the method employed), I also single out the contribution of ability. As it turns out, including a measure of ability in the model, or a proxy for it, significantly reduces the overall effect of the returns to schooling. This finding is further supported in a chapter fully dedicated to the issue of natural experiments, or so-called twin studies. Here, only a subset of identical twins is used, for which the innate ability is assumed to be equal. When analyzing an entirely new dataset comprised of only identical twins, I conclude the returns to schooling drop even further, by up to two to three percentage points. With these findings, I altogether aim to bring a compelling argument to the field that ability bias is a significant issue in the literature and should be accounted for in future research.

The present thesis is structured as follows: Chapter 2 delves into the existing literature and theoretical background of the topic. Chapter 3 describes the data collection process and explores the described data. Chapter 4 is dedicated to detecting publication bias through means of both standard and novel methods. Chapter 5 focuses on heterogeneity through means of model averaging. Chapter 6 presents calculations of the best-practice effect in the literature. Chapter 7 explores the issue of ability bias in the context of natural studies. Chapter 8 concludes the thesis.