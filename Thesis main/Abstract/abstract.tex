%-----<<<<<<<<<<<< START >>>>>>>>>>>>-----
\documentclass [a4paper,12pt]{report}  
\usepackage[a-2u]{pdfx}      		
\usepackage[cp1250]{inputenc}													
\usepackage{lmodern}
\usepackage[T1]{fontenc}
\usepackage{textcomp}
\usepackage{setspace}
\onehalfspacing

\begin{document}

\section*{Abstract}
The literature on returns to schooling is among the largest and most scrutinised fields in economics, yet it has historically overlooked the significance of ability bias. To this date, there is no consensus on the extent to which general intelligence influences the estimated returns to schooling. In this work, I assemble a dataset containing 1754 from 154 studies, and attempt to systematically analyse the role of ability in the context of the Mincer equation. I first check for publication bias, and find that controlling for it lowers the returns to education by roughly one percentage point to roughly 6-7\%. Then, using model averaging, I identify 19 highly relevant variables that influence the effect, including education type, education level, gender, or the estimation method. Lastly, in the context of twin studies  where ability is assumed identical, I construct a new dataset comprised of 154 estimates across 13 studies, and find that the influence of education drops even further, to a staggering 4-6\% returns to an additional year of schooling.
\end{document}
